
% -- classes do documento --
\documentclass[
	% -- classe de construcao --
		12pt, % tamanho da fonte
		%openright % capitulos começam em pag impar (insere pag vazia caso preciso)
		oneside, % impressao em apenas um lado
		a4paper, % formato do papel: A4
		article, % classe do documento: Artigo
	% -- opcoes da classe abntex2 --
		chapter=TITLE, % titulos de capitulos convertidos em letras maiusculas
		section=TITLE, % titulos de secoes convertidos em letras maiusculas
		subsection=TITLE, % titulos de subsecoes convertidos em letras maiusculas
	% -- opções do pacote babel --
		english, % idioma adicional para hifenizacao
		spanish, % idioma adicional para hifenizacao
		brazil % o ultimo idioma e o principal do documento
]{abntex2} % padrao ABNT2


% -- incusao de pacotes --
\usepackage[utf8]{inputenc} % codificacao do documento para leitura de caracteres unicode
\usepackage[brazil]{babel} % interpretacao do idioma pt-br
\usepackage{lmodern} % fonte Latin Modern
\usepackage{indentfirst} % indenta o primeiro paragrafo de cada secao
\usepackage{microtype} % para melhorias de justificação
\usepackage{color} % controle das cores
\usepackage[T1]{fontenc} % selecao de codigos de fonte
\usepackage{minted} % destaque de cores para codigos
\usepackage{amsmath, amsfonts, calc, xcolor} % inclusao de formulas matematicas
\usepackage{graphicx} % inclusao de graficos e imagens
\usepackage[brazilian, hyperpageref]{backref} % paginas de citacoes na bibl
\usepackage[alf, abnt-etal-list=0, abnt-etal-cite=1]{abntex2cite} % citacoes padrao ABNT
\usepackage{lipsum} % geracao de texto Lorem Ipsum
\usepackage{lastpage} % usado pela ficha catalografica


% -- configuracoes de estilos de capitulos --
\numberwithin{equation}{section}
\numberwithin{figure}{section}
\numberwithin{table}{section}
\linespread{0.95}

\makeatletter
	\setlength{\@fptop}{5pt} % posicionar figuras e tabelas no topo da pag quando adicionadas sozinhas em um pag em branco
\makeatother

\renewcommand{\ABNTEXpartfontsize}{\normalsize}
\renewcommand{\ABNTEXchapterfontsize}{\normalsize}
\renewcommand{\ABNTEXsectionfontsize}{\normalsize}
\renewcommand{\ABNTEXsubsectionfontsize}{\normalsize}

\renewcommand{\familydefault}{\sfdefault}
\renewcommand{\rmdefault}{phv}
\renewcommand{\sfdefault}{phv}
\renewcommand{\ttdefault}{pcr}

\renewcommand{\backrefpagesname}{Citado na(s) página(s):~}
\renewcommand{\backref}{}
\renewcommand*{\backrefalt}[4]{
	\ifcase #1
		Nenhuma citação no texto.
	\or
		Citado na página #2.
	\else
		Citado #1 vezes nas páginas #2.
	\fi
}

\newcommand{\quadroname}{Quadro}
\newcommand{\listofquadrosname}{Lista de quadros}

%\newfloat[chapter]{quadro}{loq}{\quadroname}
\newlistof{listofquadros}{loq}{\listofquadrosname}
\newlistentry{quadro}{loq}{0}

\setfloatadjustment{quadro}{\centering}
\counterwithout{quadro}{chapter}
\renewcommand{\cftquadroname}{\quadroname\space} 
\renewcommand*{\cftquadroaftersnum}{\hfill--\hfill}

\setfloatlocations{quadro}{hbtp}

% -- espacamento entre linhas e paragrafos --
\setlength{\parindent}{1.3cm}
\setlength{\parskip}{0.2cm}

% -- compila o indice --
\makeindex

% -- seneciona raiz do caminho para elementos graficos --
\graphicspath{ {./img/} }


% -- informacoes da capa e folha de rosto --
\titulo{Implementação do Sistema Eleitoral Brasileiro na Programação}
\autor{
Álvaro D. S. Alves \\
Elder Storck
}
\local{Goiabeiras - Vitória - ES - Brasil}
\data{2022, Fevereiro}
\orientador{João Paulo A. Almeida}
\instituicao{
Universidade Federal do Espírito Santo
\par
Curso de Engenharia de Computação
\par
Disciplina de Programação III
}
\tipotrabalho{Trabalho de Programação}
\preambulo{Implementação do Sistema Eleitoral Brasileiro na Linguaguem Java (Linguaguem de Programação Orientada a Objetos)}


% -- configuracoes do PDF --
\definecolor{blue}{RGB}{41, 5, 195}
\makeatletter
	\hypersetup{
		%pagebackref=true,
		pdftitle={\@title}, 
		pdfauthor={\@author},
		pdfcreator={Álvaro Davi Santos Alves},
		pdfkeywords={abnt}{latex}{abntex}{abntex2}{trabalho acadêmico}{programação}{java}{POO}, 
		colorlinks=true, % links coloridos
		linkcolor=blue, % cor de links internos
		citecolor=blue, % cor de citacoes bibliograficas
		filecolor=magenta,
		urlcolor=blue,
		bookmarksdepth=4
	}
\makeatother


% -- inicio do documento --
\begin{document}
	\selectlanguage{brazil}

	\frenchspacing % retirar espaço extra obsoleto entre as frases

	\phantompart % adicionar espaco para sumario

	% ----------------------------------------------------------
	% ELEMENTOS PRE-TEXTUAIS
	% ----------------------------------------------------------

	%\maketitle % criar o titulo
	\imprimircapa

	\imprimirfolhaderosto* % (o * indica que havera a ficha bibliografica)
	\newpage

	\tableofcontents % sumario
	\newpage

	% ----------------------------------------------------------
	% ELEMENTOS TEXTUAIS
	% ----------------------------------------------------------
	\textual

	\section{Introdução}

		Este trabalho tem como objetivo pôr em prática os conceitos de \emph{Programação Orientada a Objetos} (POO) utilizando a linguagem de programação \textbf{Java}.
		\\

		A tarefa em questão é utilizar as caracterísicas da POO (encapsulamento, herança, polimorfismo e instanciação) a fim de representar, de maneira abstrata porém clara, o sistema eleitoral brasileiro.

		O trabalho também se utiliza de conceitos dos paradigmas de programação anteriores, como os paradigmas da programação procedural e estruturada, ao implementar funções em blocos, estruturas de repetição, estruturas condicionais, tratamento de erros e exceções além de estruturas de dados.

	\newpage

	\section{Implementação}

    	Para a implementação da solução, o primeiro passo realizado foi o de dividir o problema em partes, por exemplo:
        \begin{itemize}
            \item Ler o arquivo csv e armazenar os dados em seus campos no seu formato correto;
            \item Implementar uma estrutura de dados que atendesse às necessidades para gerenciar os dados em questão;
            \item Criar um esquema de classes claro e objetivo;
            \item Implementar funcionalidades para gerenciar os dados;
            \item Implementar funcionalidades para exibir os relatórios;
        \end{itemize}
        Com isso surgiram classes como:
        \begin{itemize}
            \item A classe principal \verb|Main|, que apenas recebe os parâmetros e instancia as outras classes;
            \item A classe \verb|Election|, que realiza todas as ações concernentes à eleição de modo geral;
            \item A classe \verb|Utils|, que fornece funções simples de leitura e comparação;
            \item A classe \verb|Reports|, que exibe os relatórios;
            \item E, por fim, as classes \verb|Candidate| e \verb|Party|, que servem para a instanciação das entidades dos candidatos e partidos.
        \end{itemize}
        Além de outras classes utilizadas para comparação e ordenação.

        Todas as classes que possuem \textit{atributos} utilizam o recurso de \underline{encapsulamento}, visando maior restrição e controle sobre tais atributos e possuindo, além de \textit{métodos construtores}, métodos de manipulação dos atributos, conhecidos como \textit{getters} e \textit{setters}.

        Uma vez observado que as classes possuírem diferenças em demasia e poucas semelhanças, não foi vista como necessária a utilização do recurso de \underline{herança} de classes. Porém a implementação de \textit{interfaces} e o \underline{polimorfismo} foram utilizados nas classes comparadoras.

        \newpage

        \subsection{Exemplo de Classe em Java}
		\begin{minted}[mathescape, linenos, numbersep=5pt, gobble=2, frame=lines, framesep=2mm]{java}
	    // exports and imports
        package src.classes;
        import java.util.LinkedList;


        public class Party {
        	// attributes
        	private int number = 0;
        	private String name = "";
        	private String alias = "";
        	private int partyVotes = 0;
        	private int totalVotes = 0;
        	private LinkedList<Candidate> candidates = null;


        	// constructor method
        	public Party(String alias) {
        		this.alias = alias;
        	}


        	// methods
        	public int getNumber() {
        		return number;
        	}
        	public void setNumber(int number) {
        		this.number = number;
        	}

        	public String getName() {
        		return name;
        	}
        	public void setName(String name) {
        		this.name = name;
        	}

        	public int getPartyVotes() {
        		return partyVotes;
        	}
        	public void setPartyVotes(int votes) {
        		this.partyVotes = votes;
        	}
        }
		\end{minted}

    \newpage

	\section{Relacionamento entre Classes}

        O relacionamento entre as classes e interfaces mencionadas acima é representado no diagrama a seguir, descrito em UML, a \emph{Unified Modeling Language}.
    	\\

    	\centerline{
    	    \includegraphics[width=450px]{UML Classes Diagram.drawio.png}
    	}
		\centerline{
    		\href{https://github.com/AlvaroDavi5/Brazilian_Electoral_System/blob/dev/docs/img/UM_ Classes_Diagram.png}{Figura 1.0 - Diagrama de Classes em UML}
    	}

        Onde as setas tracejadas represantam a implementação das interfaces (em azul) e as linhas contínuas representam tanto a instanciação de uma classe dentro de outra classe quanto o fluxo do programa.

        Cada classe possui atributos (em amarelo) e/ou métodos (em roxo). É notório o que a classe \textbf{Election} possui uma maior quantidade de métodos do que as outras classes, isso se deve pelo fato desta classe ser referente a todo o contexto do programa para a eleição, que envolve desde a criação das entidades até a manipulação das estruturas que as contém.

        Novamente é visível que a classe \textbf{Main} apenas instancia a classe de eleição e a utiliza para obter os dados que serão exibidos na classe \textbf{Reports}.

    \newpage

	\section{Tratamento de Erros e Exceções}

        A fim de evitar erros, foram feitos tratamentos de excessões. A exemplo, para a funções de leitura e escrita (se necessário) de arquivos foi feito o tratamento de input/output. Para formatação de datas foi utilizada uma biblioteca que retornava uma exceção de conversão, tal exceção também foi tratada.

        Além disso, também foram feitos tratamentos de dados retornados por funções, a fim de evitar erros de tipagem, parâmetros, argumentos ou métodos inválidos.


	\section{Known Bugs}

        Como já dito anteriormente, o tratamento de possíveis erros também foi realizado, poŕem isso não impede que novos erros ocorram em caso de divergências, mesmo que sutis.

        Um exemplo que vale ser mencionado se encontra na linha 119 do arquivo \verb|src/Election.java|, onde os dados das linhas dos arquivos CSV são utilizado apenas caso sigam o formato padrão de separação por vírgula e caso cada linha possua a mesma quantidade de colunas/dados entre vírgulas que a primeira linha. Tal situação também possui um tratamento de exceções, que leva em conta a quantidade de colunas ser zero, a linha ser uma string vazia ou a linha ser nula. Qualquer caso fora do formato padrão ou destas 3 exceções resultaria em um erro visível no standart output, prejudicando o fluxo do programa.

        Outro exemplo relacionado ao anterior se encontra na função readFile do arquivo \verb|src/Utils.java|, que apenas leva em conta o formato padrão dos arquivos .csv para as duas entidades, candidato e partido. Qualquer tipo de dado ou formato além desses na base de dados resultaria em um bug.


	\section{Testes}

        Fora os bugs mencionados anteriormente, nenhum outro problema foi detectado durante os testes. Testes esses realizados utilizando bases de dados de exemplo, como a base de dados do município da Serra, que pode ser encontrada junto ao repositório do código no \href{https://github.com/AlvaroDavi5/Brazilian_Electoral_System}{GitHub}.

        Além disso, o trabalho passou em todos os cases de teste fornecidos pelo professor, o que auxilia a mensurar o estágio do desenvolvimento.

    \newpage

	\section{Conclusão}

        Por fim, com a realização desse trabalho foi possível compreender mais acerca da aplicabilidade do paradigma de Orientação a Objetos, pôr em prática os conhecimentos na linguagem Java, entender melhor o funcionamento do sistema eleitoral brasileiro além de adquirir contato com o processo de desenvolvimento de software de uma forma organizada, eficiente, documentada e clara, apesar de abstrata.

\end{document}
